% Alle Optionen, die an `tui-algo-seminar` übergeben werden, werden an lipics-v2021 weitergereicht.
% Lesen Sie die Dokumentation für lipics-v2021 für weitere Informationen:
% https://submission.dagstuhl.de/series/details/5#author
\documentclass[a4paper,ngerman]{tui-algo-seminar}

% Nach Belieben (de-)kommentieren Sie die folgenden Zeilen:
\usepackage{graphicx}  % Grafiken einbinden
\usepackage{algorithm2e}  % Pseudocode-Listings
\usepackage{booktabs}  % ansprechendere Tabellen
\usepackage{tikz}  % Abbildungen zeichnen
\usepackage{hyperref} % URL

\newcommand{\content}{Kreiseinzelmeisterschaft der Jugend im Ilm-Kreis 2023 in den Altersklassen U10 bis U18}
\seminar{\content}

\semester{13.11.2023}
\title{\content}
\author{Erik Skopp}

% Kommentieren Sie die folgende Zeile aus, um die Zeilennummern zu entfernen.
% Zeilennummern müssen in jeder Vorabversion enthalten sein.
% \nolinenumbers  % Diese Anweisung wird vom Paket `lineno` definiert

\begin{document}
	
	\maketitle
\thispagestyle{plain} % Seitenzahl auf Seite 1 anzeigen

\begin{abstract}
Das Schachturnier "Kreiseinzelmeisterschaft der Jugend im Ilm-Kreis 2023" zog 48 Teilnehmer aus verschiedenen Teilen des Ilmkreises an. In den Altersklassen U8 bis U18 kämpften die Kinder in fünf Runden mit je 20 Minuten Bedenkzeit. Überraschende Ergebnisse und fairer Wettbewerb prägten das Turnier, bei dem die Sieger sich für die Bezirkseinzelmeisterschaft im Januar 2024 qualifizierten. Die Veranstaltung verlief reibungslos und zeichnete sich durch eine freundliche Atmosphäre aus. 
\end{abstract}

\section{Bericht}
Am 5. November 2023 begannen die Aufbauarbeiten in der Ilmenauer Sparkassenhalle pünktlich um 8 Uhr. Eine engagierte Gruppe von Helfern sorgte dafür, dass alles für das bevorstehende Schachturnier vorbereitet wurde. 48 aufgeregte Kinder aus verschiedenen Teilen des Ilmkreises strömten voller Vorfreude in die Halle, gespannt darauf, herauszufinden, wer von ihnen in der jeweiligen Altersklasse als bester Schachspieler hervorgehen würde.

In den Altersklassen U8, U10, U12, U14, U16 und U18 wurde mit vollem Einsatz und großer Leidenschaft um die begehrten Siege gekämpft. Fünf Runden mit je 20 Minuten Bedenkzeit pro Kind und Runde sorgten für spannende Partien, die mitunter bis zu 40 Minuten dauerten. Die jüngsten Teilnehmer hatten gelegentlich Schwierigkeiten, die gesamte Bedenkzeit zu nutzen, während in den Altersklassen U16 und U18 oft bis zur letzten Sekunde um jeden Zug gerungen wurde.

Kurz vor halb neun füllte sich die Halle mit den ersten Teilnehmern und ihren begleitenden Eltern, die sich registrierten und die aufkommende Aufregung spürten. Die U10 stellte mit 20 Kindern aus drei Vereinen die stärkste Gruppe. Sieben Kinder kamen aus Ilmenau und sieben aus BW-Stadtilm. Trotz der Tatsache, dass in kaum einer Runde die vollen 20 Minuten Bedenkzeit ausgeschöpft wurden, war der Kampfgeist der Kinder ungebrochen.

Ray Kashvi vom Ilmenauer SV sicherte sich den ersten Platz in der U10 mit 4 von 5 Punkten. Auf Platz 2 landete Erik Borchert (SV Stützerbach) mit ebenfalls 4 aus 5 Punkten, wobei die Buchholzwertung den Ausschlag für Kashvi gab. Den dritten Platz belegte Ben Kempf aus BW-Stadtilm. Diese spannenden Partien unterstreichen, dass in der U10 das Alter nicht ausschlaggebend für das schachliche Können ist.

In der U12 traten nur 9 Spieler an, wobei der Ilmenauer SV drei Kinder und BW-Stadtilm fünf Kinder stellte. Die ungerade Spieleranzahl führte dazu, dass immer jemand spielfrei hatte. In der U10 wagten sich erstmals Spieler mit einer Deutschen Wertungszahl in den Wettkampf. Raphael Zentgraf konnte das Turnier ohne Verluste für sich entscheiden und sicherte sich den ersten Platz mit 4,5 aus 5 Punkten. Paul Hübner erreichte Platz 2 mit 4 aus 5 Punkten, gefolgt von Anna Wäldchen vom Ilmenauer SV auf Platz 3.

Die U14, bestehend aus nur 8 Teilnehmern, sah eine spannende Konkurrenz. Richter Finjas (1025) von BW-Stadtilm und Lehmann Norik (888) vom Ilmenauer SV bildeten die Spitze. Am Ende konnte sich Lehmann Norik mit 5 aus 5 Punkten klar durchsetzen und alle Runden für sich entscheiden. Richter Finjas erreichte Platz 2, und der dritte Platz wurde durch eine Schiedsrichterentscheidung getroffen.

Die Altersklassen U16 und U18 spielten aufgrund geringer Spielerzahlen gemeinsam. Mit insgesamt 8 Teilnehmern gab es keine spielfreien Runden. Diese Gruppe brachte viele lange und hochkarätige Spiele hervor, in denen oft nur noch weniger als zehn Sekunden auf der Uhr waren. Markus Kaiser (1638) von SG Arnstadt-Stadtilm konnte sich mit 4,5 Punkten den ersten Platz sichern. Hanna Görlach aus Ilmenau erreichte überraschend Platz 2 mit 3,5 Punkten, gefolgt von Ivan Krasnov auf Platz 3.

Das Turnier begann um 9 Uhr und endete gegen 16:00 Uhr, wobei viele strahlende Gewinner hervorgingen. Die ersten drei jeder Altersklasse haben sich für die Bezirkseinzelmeisterschaft im Januar 2024 bei Blau Weiß Stadtilm qualifiziert. Das Turnier verlief reibungslos und fair und es war ein sportliches Event, das im nächsten Jahr sicherlich wiederholt werden sollte.


Erik Skopp



\section{Bilder}
Die Bilder finden Sie in der Cloud des Ilmenauer Schachvereines: 
\begin{itemize}
	\item[-]: \url{https://cloud.ilmenauer-schachverein.de}
\end{itemize}

\section{Chess-Results}
\begin{itemize}
	\item[-] U10: \url{https://chess-results.com/tnr844671.aspx?lan=1}
	\item[-] U12: \url{https://chess-results.com/tnr844674.aspx?lan=1}
	\item[-] U14: \url{https://chess-results.com/tnr844679.aspx?lan=1}
	\item[-] U16 / U18: \url{https://chess-results.com/tnr844680.aspx?lan=1}
\end{itemize}
\clearpage

\section{Tabellen}
\subsection{U10}
	
	\begin{center}
		\begin{tabular}{|c|c|l|c|l|c|c|c|c|}
			\hline
			\textbf{Rk.} & \textbf{SNo} & \textbf{Name} & \textbf{Rtg} & \textbf{Club/City} & \textbf{Pts.} & \textbf{TB1} & \textbf{TB2} & \textbf{TB3} \\
			\hline
			1 & 13 & Ray, Kashvi & 0 & Ilmenauer SV & 4 & 14,5 & 11,00 & 3 \\
			2 & 9 & Bochert, Erik & 0 & SV Stützerbach & 4 & 13 & 10,00 & 4 \\
			3 & 6 & Kempf, Ben & 0 & SG Blau-Weiß Stadtilm & 4 & 12,5 & 10,25 & 3 \\
			4 & 15 & Al-Sayeh, Bassam & 0 & Ilmenauer SV & 3,5 & 15 & 9,00 & 3 \\
			5 & 4 & Nicolai, Franz & 0 & SG Blau-Weiß Stadtilm & 3,5 & 12,5 & 6,50 & 3 \\
			6 & 7 & Linke, Eddy & 0 & SG Blau-Weiß Stadtilm & 3 & 15 & 7,00 & 3 \\
			7 & 2 & Hübner, Jakob & 0 & SG Blau-Weiß Stadtilm & 3 & 12 & 5,50 & 3 \\
			8 & 19 & Richter, Fabian & 0 & SG Arnstadt-Stadtilm & 3 & 11 & 6,00 & 3 \\
			9 & 8 & Nawatzki, Elenor Viktoria & 0 & SG Blau-Weiß Stadtilm & 2,5 & 14 & 5,25 & 2 \\
			10 & 12 & Schulze, Henry & 0 & SG Arnstadt-Stadtilm & 2,5 & 13,5 & 7,00 & 2 \\
			11 & 5 & Schechinger, Karl & 0 & SG Blau-Weiß Stadtilm & 2,5 & 13 & 4,75 & 2 \\
			12 & 3 & Mausolf, Richard & 0 & SG Blau-Weiß Stadtilm & 2 & 15,5 & 4,50 & 2 \\
			13 & 10 & Zeitsch, Anastasia & 0 & SG Arnstadt-Stadtilm & 2 & 13,5 & 4,50 & 2 \\
			14 & 11 & Richter, Bruno & 0 & SG Arnstadt-Stadtilm & 2 & 12,5 & 3,50 & 2 \\
			15 & 1 & Jung, Bruce & 0 & SG Blau-Weiß Stadtilm & 2 & 10,5 & 2,75 & 1 \\
			16 & 17 & Luu, Phong & 0 & Ilmenauer SV & 2 & 8 & 1,50 & 2 \\
			17 & 16 & Baghdadi, Mouhiy & 0 & Ilmenauer SV & 1,5 & 12 & 2,25 & 1 \\
			18 & 18 & Tran, Julia & 0 & Ilmenauer SV & 1,5 & 10,5 & 3,00 & 1 \\
			19 & 20 & Mona, Farah & 0 & Ilmenauer SV & 1,5 & 9,5 & 0,75 & 1 \\
			20 & 14 & Akili, Bader Aldyn & 0 & Ilmenauer SV & 0 & 12 & 0,00 & 0 \\
			\hline
		\end{tabular}
	\end{center}
	

\subsection{U12}
\begin{center}
	\begin{tabular}{|c|c|l|c|l|c|c|c|c|}
		\hline
		\textbf{Rk.} & \textbf{SNo} & \textbf{Name} & \textbf{Rtg} & \textbf{Club/City} & \textbf{Pts.} & \textbf{TB1} & \textbf{TB2} & \textbf{TB3} \\
		\hline
		1 & 1 & Zentgraf, Raphael & 779 & SG Blau-Weiß Stadtilm & 4,5 & 14,5 & 13,25 & 4 \\
		2 & 3 & Hübner, Paul & 0 & SG Blau-Weiß Stadtilm & 4 & 12,5 & 8,00 & 4 \\
		3 & 7 & Wäldchen, Anna & 0 & Ilmenauer SV & 3 & 12,5 & 4,00 & 3 \\
		4 & 6 & Richter, Theodor & 0 & SG Arnstadt-Stadtilm & 3 & 11,5 & 4,50 & 3 \\
		5 & 2 & Möller, Lia Sofie & 747 & SG Blau-Weiß Stadtilm & 3 & 11 & 5,00 & 3 \\
		6 & 5 & Rak, Maik & 761 & SG Blau-Weiß Stadtilm & 2,5 & 13,5 & 4,25 & 2 \\
		7 & 9 & Nasiri, Ronika & 0 & Ilmenauer SV & 2 & 14 & 4,50 & 2 \\
		8 & 4 & Kanzler, Torsten Hermann & 0 & SG Blau-Weiß Stadtilm & 2 & 10,5 & 1,50 & 2 \\
		9 & 8 & Mona, Omar & 0 & Ilmnenauer SV & 1 & 11 & 1,50 & 1 \\
		\hline
	\end{tabular}
\end{center}

\subsection{U14}
\begin{center}
	\begin{tabular}{|c|c|l|c|c|c|c|c|c|c|c|}
		\hline
		\textbf{Rk.} & \textbf{SNo} & \textbf{Name} & \textbf{Gr} & \textbf{FED} & \textbf{Rtg} & \textbf{RtgN} & \textbf{Club/City} & \textbf{Pts.} & \textbf{TB1} & \textbf{TB2} \\
		\hline
		1 & 5 & Lehmann, Norik & & - & 886 & 886 & Ilmenauer SV & 5 & 12,00 & 5 \\
		2 & 1 & Richter, Finjas & & - & 1025 & 1025 & SG Blau-Weiß Stadtilm & 4 & 13,00 & 4 \\
		3 & 2 & Bochert, Marius & & - & 0 & 0 & SV Stützerbach & 3 & 13,00 & 3 \\
		4 & 4 & Schnur, Oskar & & - & 0 & 0 & SV Stützerbach & 3 & 13,00 & 3 \\
		5 & 7 & Jacobi, Neil Evan & & - & 0 & 0 & Ilmenauer SV & 2 & 9,00 & 2 \\
		6 & 6 & Song, Tony Chuhan & & - & 0 & 0 & Ilmenauer SV & 1 & 15,00 & 1 \\
		7 & 8 & Nguyen, Leon & & & 0 & 0 & Ilmenauer SV & 1 & 15,00 & 1 \\
		8 & 3 & Handschuh, Karl & & - & 0 & 0 & SV Stützerbach & 1 & 10,00 & 1 \\
		\hline
	\end{tabular}
\end{center}


\subsection{U16 / U18}
	\begin{center}
	\begin{tabular}{|c|c|l|c|c|c|c|c|c|c|}
		\hline
		\textbf{Rk.} & \textbf{SNo} & \textbf{Name} & \textbf{sex} & \textbf{RtgN} & \textbf{Club/City} & \textbf{Pts.} & \textbf{TB1} & \textbf{TB2} & \textbf{TB3} \\
		\hline
		1 & 5 & Kaiser, Markus & U18 & 1638 & SG Arnstadt-Stadtilm & 4,5 & 12,5 & 10,75 & 4 \\
		2 & 8 & Görlach, Hanna Pauline & U18W & 0 & Ilmenauer SV & 3,5 & 13 & 7,75 & 3 \\
		3 & 7 & Krasnov, Ivan & & 952 & Ilmenauer SV & 3 & 14 & 7,50 & 2 \\
		4 & 6 & Feuerpfeil, Tyler Joel & & 0 & SG Arnstadt-Stadtilm & 2,5 & 14 & 4,50 & 2 \\
		5 & 2 & König, Cornelius & & 0 & SG Blau-Weiß Stadtilm & 2 & 11,5 & 2,50 & 2 \\
		6 & 3 & Kirsch, Valentin & & 991 & SG Blau-Weiß Stadtilm & 2 & 11 & 3,50 & 2 \\
		7 & 1 & Keßler, Finjas Ephraim & & 1046 & SG Blau-Weiß Stadtilm & 1,5 & 12,5 & 2,50 & 1 \\
		8 & 4 & Hahn, Thorwald & & 0 & TSV 1886 Geschwenda & 1 & 11,5 & 2,00 & 1 \\
		\hline
	\end{tabular}
\end{center}

\end{document}

