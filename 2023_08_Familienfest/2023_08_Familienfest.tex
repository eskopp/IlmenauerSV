% Alle Optionen, die an `tui-algo-seminar` übergeben werden, werden an lipics-v2021 weitergereicht.
% Lesen Sie die Dokumentation für lipics-v2021 für weitere Informationen:
% https://submission.dagstuhl.de/series/details/5#author
\documentclass[a4paper,ngerman]{tui-algo-seminar}

% Nach Belieben (de-)kommentieren Sie die folgenden Zeilen:
\usepackage{graphicx}  % Grafiken einbinden
\usepackage{algorithm2e}  % Pseudocode-Listings
\usepackage{booktabs}  % ansprechendere Tabellen
\usepackage{tikz}  % Abbildungen zeichnen
\usepackage{hyperref} % URLS

\seminar{Ilmenauer Familienfest 2023}

\semester{26.08.2023}
\title{Ilmenauer Familienfest 2023}
\author{Erik Skopp}

% Kommentieren Sie die folgende Zeile aus, um die Zeilennummern zu entfernen.
% Zeilennummern müssen in jeder Vorabversion enthalten sein.
% \nolinenumbers  % Diese Anweisung wird vom Paket `lineno` definiert
\nolinenumbers
\begin{document}

\maketitle

\begin{abstract}
Bericht über das Ilmenauer Familienfest mit integriertem Turnier vom 26.08.2023
\end{abstract}

\section{Bericht}
Am 26. August 2023 fand im Schülerfreizeitzentrum Ilmenau ein aufregendes Schachturnier statt, begleitet von einer fröhlichen Abschlussfeier. Das Turnier wurde im Schweizer System mit 5 Runden ausgetragen, wobei jedem Spieler pro Partie 15 Minuten zur Verfügung standen, was insgesamt 30 Minuten pro Partie ergab. Insgesamt nahmen 9 Teams teil, wobei jedes Team aus 2 Schachspielern bestand, wobei es keine Einschränkung gab, ob sie Familienmitglieder waren oder nicht. Es sei angemerkt, dass Nicht-Vereinsmitglieder bei einem Sieg 2 Punkte statt nur einem erhielten, um Fairness zu gewährleisten.\\
In der Familienwertung sicherte sich Familie Handschuh den ersten Platz, gefolgt von Familie Schnur auf dem zweiten und Familie Ray auf dem dritten Platz. Familie Bochert und Familie Kransnov belegten die Plätze 4 und 5 in der Familienwertung. In der Teamwertung gewann das Duo Erik Skopp und Wolfgang Brandt den ersten Platz, gefolgt von Norik Lehmann und Frank Winger auf dem zweiten Platz sowie Chau und Hanna Görlach auf dem dritten Platz. Die ersten drei Teams erhielten Pokale, und alle Teilnehmer wurden mit Sachpreisen belohnt.\\
Für das leibliche Wohl sorgte den ganzen Tag über ein Grill, der für jeden Geschmack das passende Essen bot. Gegen 14 Uhr endete das Turnier, und die Teilnehmer genossen verschiedene Spiele und geselliges Beisammensein, darunter Wikingerschach, Vier gewinnt und mehr. Das Schülerfreizeitzentrum bot auch verschiedene Aktivitäten wie Slacklining, Tierbeobachtungen, Tischtennis und gemütliche Lagerfeuer.\\
Wolfgang Brandt brachte einen Volleyball mit, der für spannende Spiele sorgte. Abends wurde ein köstliches Abendessen mit veganen Gemüse- und Grillgerichten serviert. Wie bei solchen Veranstaltungen üblich, gab es auch einige Runden Werwolf. Als die Nacht hereinbrach, genossen die Teilnehmer Stockbrot am Lagerfeuer, und der Abend klang in gemütlicher Runde aus.\\
Ein herzlicher Dank geht an alle Helfer und Teilnehmer, die diesen wundervollen Tag möglich gemacht haben.\\ 
\\
Erik Skopp

\section{Anhang}
\begin{itemize}
    \item Bilder: \url{https://cloud.ilmenauer-schachverein.de/s/pNQHggDA5xWCsK7} - Stand 01.09.2023
\end{itemize}
\end{document}