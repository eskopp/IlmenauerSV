% Alle Optionen, die an `tui-algo-seminar` übergeben werden, werden an lipics-v2021 weitergereicht.
% Lesen Sie die Dokumentation für lipics-v2021 für weitere Informationen:
% https://submission.dagstuhl.de/series/details/5#author
\documentclass[a4paper,ngerman]{tui-algo-seminar}

% Nach Belieben (de-)kommentieren Sie die folgenden Zeilen:
\usepackage{graphicx}  % Grafiken einbinden
\usepackage{algorithm2e}  % Pseudocode-Listings
\usepackage{booktabs}  % ansprechendere Tabellen
\usepackage{tikz}  % Abbildungen zeichnen
\usepackage{hyperref} % URL

\newcommand{\content}{1. Ilmenauer Halloween Blitz 2023}
\seminar{\content}

\semester{13.11.2023}
\title{\content}
\author{Erik Skopp}

% Kommentieren Sie die folgende Zeile aus, um die Zeilennummern zu entfernen.
% Zeilennummern müssen in jeder Vorabversion enthalten sein.
% \nolinenumbers  % Diese Anweisung wird vom Paket `lineno` definiert

\begin{document}
	
	\maketitle
\thispagestyle{plain} % Seitenzahl auf Seite 1 anzeigen

\begin{abstract}


. 
\end{abstract}

\section{Bericht}

Am 3. November fand im Ilmenauer Spiellokal (EAZ) mit großer Begeisterung das erste Halloween-Blitzturnier statt. Die Spannung lag in der Luft, als 21 Schachspieler aus sechs verschiedenen Nationen, darunter Deutschland, die Ukraine, Palästina, Russland und Indien, zusammenkamen, um ihre Fähigkeiten in einem internationalen Wettbewerb unter Beweis zu stellen.

Die Atmosphäre war einladend, unabhängig von der Spielerfahrung – vom Anfänger bis zum Profi. Gespielt wurden 11 Runden im bewährten Schweizer System, wobei eine Bedenkzeit von 5 Minuten ohne Inkrement galt. Dieses nationale Turnier war geprägt von einer vielfältigen Teilnehmergruppe, die die Leidenschaft für das königliche Spiel teilte.

Die Spitzengruppe setzte sich aus herausragenden Spielern wie Matthias Liedmann (Ilmenauer SV), Marco Geißhirt (SG Barchfeld/Breitungen) und Pascal Eichenauer (SG Würzburg) zusammen. Die ersten Runden zeichneten sich durch eine solide Punkteausbeute aus, obwohl Marco zu Beginn etwas strauchelte und aus den ersten drei Partien nur 1,5 Punkte holte. Dies ermöglichte Pascal, mit einem halben Punkt Vorsprung die Führung zu übernehmen.

Ainur (Ilmenauer SV) stieg erst in der dritten Runde ein und kämpfte sich von hinten nach vorne. In Runde 6 etablierte sich Stefan Schenk mit soliden 4 von 6 Punkten auf dem dritten Platz, den er bis zur achten Runde hielt. Dort traf er auf Marco und musste eine Niederlage einstecken. Aufgrund von vorzeitigen Abgängen waren zu diesem Zeitpunkt immer einige Spieler spielfrei.

Die Nachwuchsspieler des Ilmenauer Schachvereins, Hanna und Norik, sammelten ebenfalls solide Punkte und zeigten ihre Stärke in einem wettbewerbsintensiven Feld. In Runde neun trafen zwei Favoriten am ersten Brett aufeinander: Marco gegen Ainur, während Pascal nach dem Remis gegen Ainur in Runde 7 auf das zweite Brett rutschte.

Nach der 10. Runde führte Marco vor Pascal und Matthias. Ainur gewann und remisierte, doch man konnte erkennen, dass ihm die zwei fehlenden Partien zu Beginn fehlten. In der letzten Runde gelang es Leonid (SV Empor Erfurt), Marco am ersten Brett zu schlagen, während Pascal gegen Fares am zweiten Brett gewann. Dies führte zu einer Punktgleichheit zwischen Marco und Pascal. Der Buchholzunterschied von +0,5 sicherte schließlich Marco den ersten Platz und Pascal den zweiten. Auf dem dritten Platz landete Matthias, gefolgt von Ainur.

Das Turnier war nicht nur sportlich erfolgreich, sondern auch ein gesellschaftlicher Erfolg. 22 Schachspieler aus verschiedenen Teilen der Welt versammelten sich, um ihrer gemeinsamen Leidenschaft zu frönen. Angesichts dieses positiven Feedbacks planen wir bereits ein weiteres Blitzturnier zum Nikolaus, zu dem alle Schachspieler herzlich eingeladen sind.

Erik Skopp

\section{Bilder}
Die Bilder finden Sie in der Cloud des Ilmenauer Schachvereines: 
\begin{itemize}
	\item[-]: \url{https://cloud.ilmenauer-schachverein.de}
\end{itemize}

\section{Chess-Results}
\begin{itemize}
	\item[-] https://chess-results.com/tnr839981.aspx?lan=1
\end{itemize}

\section{Tabellen}
	\begin{center}
	\begin{tabular}{|c|c|l|c|c|c|c|c|c|c|}
		\hline
		\textbf{Rk.} & \textbf{SNo} & \textbf{Name} & \textbf{FED} & \textbf{Rtg} & \textbf{Club/City} & \textbf{Pts.} & \textbf{TB1} & \textbf{TB2} & \textbf{TB3} \\
		\hline
		1 & 3 & Geißhirt, Marco & GER & 1998 & SG Barchfeld/Breitungen & 8,5 & 70 & 54,00 & 8 \\
		2 & 1 & Eichenauer, Pascal & GER & 2097 & SV Würzburg von 1865 e.V. & 8,5 & 67 & 47,25 & 8 \\
		3 & 2 & Liedmann, Matthias & GER & 2050 & Ilmenauer SV & 7,5 & 73,5 & 45,00 & 7 \\
		4 & 22 & Ziganshin, Ainur & RUS & 2198 & Ilmenauer SV & 7,5 & 66,5 & 42,25 & 7 \\
		5 & 5 & Elkhanov, Leonid & GER & 1813 & SV Empor Erfurt & 7 & 73 & 42,50 & 7 \\
		6 & 8 & Dudeja, Iresh & IND & 1586 & Ilmenauer SV & 7 & 70 & 42,75 & 6 \\
		7 & 9 & Hartung, Markus & GER & 1584 & Ilmenauer SV & 6 & 71,5 & 31,75 & 5 \\
		8 & 4 & Schenk, Stefan & GER & 1909 & Ilmenauer SV & 6 & 68 & 30,50 & 6 \\
		9 & 12 & Eisenbach, Markus Dr. & GER & 1404 & Ilmenauer SV & 6 & 55,5 & 21,50 & 6 \\
		10 & 7 & Lehmann, Georg & GER & 1586 & ESV Lok Meiningen & 6 & 45,5 & 20,50 & 6 \\
		11 & 10 & Skopp, Erik & GER & 1561 & Ilmenauer SV & 5,5 & 67,5 & 29,75 & 5 \\
		12 & 21 & Schvachko, Mark & UKR & 0 & & 5,5 & 59 & 22,25 & 5 \\
		13 & 15 & Abuawad, Fares & PSE & 0 & Ilmenauer SV & 5 & 65 & 22,00 & 5 \\
		14 & 19 & Jung, Timo & GER & 0 & & 5 & 56 & 18,50 & 5 \\
		15 & 13 & Lehmann, Norik & GER & 886 & Ilmenauer SV & 5 & 51,5 & 14,00 & 5 \\
		16 & 6 & Michael, Torsten & GER & 1680 & Ilmenauer SV & 4 & 57,5 & 13,50 & 4 \\
		17 & 17 & Görlach, Hanna Pauline & GER & 0 & Ilmenauer SV & 4 & 46,5 & 13,50 & 4 \\
		18 & 11 & Böhmer, Leon & GER & 974 & Ilmenauer SV & 3 & 54 & 8,00 & 3 \\
		19 & 14 & Winger, Frank & GER & 838 & Ilmenauer SV & 2 & 51,5 & 6,00 & 2 \\
		20 & 18 & Greul, Simon & GER & 0 & Ilmenauer SV & 1 & 48 & 5,50 & 1 \\
		21 & 20 & Möller, Nico & GER & 0 & & 1 & 40,5 & 4,00 & 1 \\
		\hline
	\end{tabular}
\end{center}

\end{document}

