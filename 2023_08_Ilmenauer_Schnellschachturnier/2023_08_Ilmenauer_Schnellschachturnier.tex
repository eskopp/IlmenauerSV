% Alle Optionen, die an `tui-algo-seminar` übergeben werden, werden an lipics-v2021 weitergereicht.
% Lesen Sie die Dokumentation für lipics-v2021 für weitere Informationen:
% https://submission.dagstuhl.de/series/details/5#author
\documentclass[a4paper,ngerman]{tui-algo-seminar}

% Nach Belieben (de-)kommentieren Sie die folgenden Zeilen:
\usepackage{graphicx}  % Grafiken einbinden
\usepackage{algorithm2e}  % Pseudocode-Listings
\usepackage{booktabs}  % ansprechendere Tabellen
\usepackage{tikz}  % Abbildungen zeichnen
\usepackage{hyperref} % URL

\seminar{29. Ilmenauer Schnellschach Turnier 2023}

\semester{19.08.2023}
\title{29. Ilmenauer Schnellschach Turnier 2023}
\author{Erik Skopp}

% Kommentieren Sie die folgende Zeile aus, um die Zeilennummern zu entfernen.
% Zeilennummern müssen in jeder Vorabversion enthalten sein.

%\nolinenumbers  % Diese Anweisung wird vom Paket `lineno` definiert

\begin{document}

\maketitle

\begin{abstract}
Bericht über das Ilmenauer Schnellschach Turnier 2023 vom 19.08.2023
\end{abstract}

\section{Bericht}
Am 19. August fand das 29. Ilmenauer Schnellschachturnier statt, wie üblich im Rathaussaal. Mit drei Titelträgern und neun Spielern, die ein ELO- oder DWZ-Rating von über 2000 hatten, war ein spannendes Turnier garantiert. Die Spitzenspieler waren IM Ference Langheinrich vom SV Empor Erfurt, IM Joachim Brüggemann vom ESK Erfurt, Ainur Ziganshin vom Ilmenauer SV und FM Christian Aepfler aus Weimar. Mit 48 Spielen war das Turnier so groß wie nie zuvor. Wir konnten Schachfreunde aus sechs verschiedenen Nationen begrüßen, nämlich Kolumbien, Deutschland, Indien, Russland, der Türkei und der Ukraine. Vom Ilmenauer SV nahmen Ainur Ziganshin, Michael Thorsten, Erik Skopp, Stefan Schenk, Iresh Dudeja, Leon Böhmer, Markus Hartung, Frank Winger, Kim-Chi Wenzel, Hanna Görlach und Ivan Krasnov teil. In sieben hart umkämpften Runden setzte sich IM Ference Langheinrich am Ende als Turniersieger durch. Die Plätze 2 bis 4 waren mit jeweils 5,5 Punkten punktgleich, und hier entschied nur die Zweitwertung über die Platzierungen. Luca Franke vom SV 1861 Liebschwitz sicherte sich mit der zweithöchsten Buchholzwertung den zweiten Platz, obwohl er ungeschlagen blieb, aber Remis gegen IM Joachim Brüggemann, IM Ferenc Langheinrich und Philipp Zitzelsberger spielte. Mit 5,5 Punkten aus 7 Runden und einer Performance von 2227 erreichte er Platz 2. Platz 3 ging an FM Christian Aepfler, der zwar nur ein Remis hatte, sich jedoch dem zweiten Platz geschlagen geben musste. Ebenfalls mit 5,5 Punkten erreichte Philipp Zitzelsberger Platz 4, wobei er ebenfalls nur eine Partie verlor und ein Remis erzielte. Die Spitzenspieler lagen so dicht beieinander, dass erst die Zweitwertung ab Platz 2 die Platzierungen bestimmte. Neben den Preisen für die Plätze 1 bis 3 wurde auch ein Heldenpokal verliehen. Dieser ging an den Inder Srinivasan Rajagopalan, der seine DWZ-Zahl virtuell um 84 Punkte verbesserte. Der beste Jugendspieler (U18) war Anh Duc Kevin Tran. Ursprünglich sollte Luca Franke diesen Preis erhalten, aber da er Zweiter im Turnier war, ging der Preis an Anh Duc Kevin. Platz 2 bei den Jugendlichen ging an Markus Kaiser, Platz 3 an Nico Franke. Der beste Ilmenauer Spieler war Ainur, der Platz 10 erreichte. Es gab während des gesamten Turniers keine schwerwiegenden Streitfälle. Das Turnier konnte nach der Siegerehrung um 18:30 Uhr erfolgreich abgeschlossen werden. Ein besonderer Dank gilt auch allen Helfern des Ilmenauer SV. Wir freuen uns, das Turnier ausrichten zu dürfen, und bedanken uns dafür, dass es so gut angenommen wurde. Bis zum nächsten Jahr.\\
\\ 
Erik Skopp
\section{Anhang}

\begin{itemize}
    \item Material (Tabellen und Bilder): 
    \begin{itemize}
        \item \url{https://cloud.ilmenauer-schachverein.de/s/wAYZZ2agaGaWqgF}
        \item obrige URL ist bis Ende 2023 aktuell 
    \end{itemize}
        \item Turnierseite
        \begin{itemize}
            \item \url{https://chess-results.com/tnr791487.aspx?lan=1}
        \end{itemize}
    \end{itemize}
\end{document}