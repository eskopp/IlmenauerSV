% Alle Optionen, die an `tui-algo-seminar` übergeben werden, werden an lipics-v2021 weitergereicht.
% Lesen Sie die Dokumentation für lipics-v2021 für weitere Informationen:
% https://submission.dagstuhl.de/series/details/5#author
\documentclass[a4paper,ngerman]{tui-algo-seminar}

% Nach Belieben (de-)kommentieren Sie die folgenden Zeilen:
\usepackage{graphicx}  % Grafiken einbinden
\usepackage{algorithm2e}  % Pseudocode-Listings
\usepackage{booktabs}  % ansprechendere Tabellen
\usepackage{tikz}  % Abbildungen zeichnen
\usepackage{hyperref} % URL

\newcommand{\content}{Regionaler-Schiedsrichter-Lehrgang des Thüringer Schachbundes im September 2023}
\seminar{\content}

\semester{16.09.2023}
\title{\content}
\author{Erik Skopp}

% Kommentieren Sie die folgende Zeile aus, um die Zeilennummern zu entfernen.
% Zeilennummern müssen in jeder Vorabversion enthalten sein.
\nolinenumbers  % Diese Anweisung wird vom Paket `lineno` definiert

\begin{document}

\maketitle

\begin{abstract}
\content
\end{abstract}

\section{Bericht}
Am 16. und 17. September wurde im Vereinsheim des SV Medizin Erfurt eine Schulung und Weiterbildung für Schiedsrichter im regionalen Bereich durchgeführt. Diese Schulung konnte auch als Fortbildung für die C-Trainer-Lizenz genutzt werden. Zusätzlich zur RSR-Lizenz wurde den Absolventen auch der Titel des FIDE-Arbiters verliehen, der offiziell dazu berechtigt, FIDE/ELO-Turniere zu leiten und auszuwerten. Drei Schachspieler des Ilmenauer SV, nämlich Markus Hartung, Markus Eisenbach und Erik Skopp, hatten sich für diese Veranstaltung angemeldet. Bedauerlicherweise konnte Markus Hartung aufgrund einer kurzfristigen Krankheit nicht am Lehrgang teilnehmen.
Der Kurs bestand aus insgesamt 20 Lehrstunden, wobei ein Schwerpunkt auf Regelkenntnissen und den FIDE-Regeln lag. Auch die Turnierordnung, Protest- und Verfahrensfragen sowie die Meldung von ELO- und DWZ-Werten wurden ausführlich behandelt. Des Weiteren wurden Themen wie die Sanktionierung von Spielern und praktische Beispiele aus dem Schiedsrichteralltag besprochen.
Nach Abschluss des Lehrgangs folgte eine Prüfung. Zum Bestehen waren 80\% erforderlich, also 80 von 100 Punkten. Die RSR-Lizenz hat eine Gültigkeitsdauer von 5 Jahren, in denen eine Weiterbildung absolviert werden muss, um die Lizenz zu verlängern. Der Kurs wurde von der Referentin Heike Goldmund, einer nationalen Schiedsrichterin, geleitet.
Der Kurs erwies sich als äußerst lehrreich, und die Teilnehmer konnten viele neue Erkenntnisse gewinnen. Ein gutes Beispiel dafür war die Erkenntnis, dass Schachpartien nicht zwangsläufig mit einem Sieg enden müssen, sondern unter bestimmten Bedingungen auch mit einem Sonderergebniss (bspw. 0-0,5) gewertet werden können. Außerdem wurden Lösungen für den Umgang mit mehreren gleichzeitigen Problemen erarbeitet. An beiden Tagen wurde das leibliche Wohl hervorragend versorgt, und wir möchten besonders Schachfreund Michael Nagel loben, der für köstliche Mahlzeiten an beiden Tagen gesorgt hat.
Am Sonntag erhielten die Teilnehmer eine Einweisung in die Handhabung von Schachuhren, gefolgt von der schriftlichen Prüfung, die 2,5 Stunden dauerte. Trotz der Hitze im Raum wurde der Kurs als äußerst interessant empfunden, und die Teilnehmer konnten viel neues Wissen aufnehmen.\\
Zum jetzigen Stand (\today) sind die Prüfungen von Markus Eisenbach und Erik Skopp noch nicht korrigiert worden. Es liegt folglich noch kein Ergebniss vor.\\
\\
Erik Skopp


\end{document}